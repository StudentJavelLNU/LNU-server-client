%------------------------------------------------------------------------------
%
% LaTeX-mall för examensarbeten vid LNU
% Skapad av Marcus Wilhelmsson, Institutionen för Datavetenskap
% Fakulteten för Teknik
% Linnéuniversitetet
%
% Licens: Creative Commons BY
%
% 
%------------------------------------------------------------------------------
%
%------------------------------------------------------------------------------
%	Inställningar och dokumentkonfiguration
%------------------------------------------------------------------------------

\documentclass[a4paper,12pt]{article} % A4-sida och 12 punkters fontstorlek


\usepackage[T1]{fontenc} % 8-bitarskodning som har 256 glyfer
\usepackage{times} % Typsnitt i dokumentet
\usepackage[swedish,english]{babel} % Svenskt språk, engelska för extra abstract
\usepackage[utf8]{inputenc} % För svenska tecken (UTF-8)
\usepackage{dtklogos} % Logos för t.ex. LaTeX, BibTeX, etc.
\usepackage{wallpaper} % Bakgrundsbild
\usepackage[absolute]{textpos} % Möjlighet att absolutpositionera text
\usepackage[top=2cm, bottom=2.5cm, left=3cm, right=3cm]{geometry} % Ställ in marginaler
\usepackage{appendix} % Stöd för separat hantering av bilagor
\usepackage{cite}
\usepackage{listings}
\usepackage[hidelinks]{hyperref}
\usepackage{float}


\setcounter{secnumdepth}{3} % Fem nivåer av underrubriksnumrering
\setcounter{tocdepth}{3} % Fem nivåer av underrubriksnumrering i innehållsförteckning

\usepackage{sectsty} % Ändra storlek på section och subsection till 12 punkter
\sectionfont{\fontsize{14}{15}\selectfont}
\subsectionfont{\fontsize{12}{15}\selectfont}
\subsubsectionfont{\fontsize{12}{15}\selectfont}

\usepackage{csquotes} % Används för att hantera citat


%------------------------------------------------------------------------------
%	Denna del används för att skapa boxen med författare, handledare, etc.

\newsavebox{\mybox}
\newlength{\mydepth}
\newlength{\myheight}

\newenvironment{sidebar}%
{\begin{lrbox}{\mybox}\begin{minipage}{\textwidth}}%
{\end{minipage}\end{lrbox}%
 \settodepth{\mydepth}{\usebox{\mybox}}%
 \settoheight{\myheight}{\usebox{\mybox}}%
 \addtolength{\myheight}{\mydepth}%
 \noindent\makebox[0pt]{\hspace{-20pt}\rule[-\mydepth]{1pt}{\myheight}}%
 \usebox{\mybox}}

%------------------------------------------------------------------------------
%	Titel-sektion
%------------------------------------------------------------------------------
\newcommand\BackgroundPic{
    \put(-2,-3){
    \includegraphics[keepaspectratio,scale=0.3]{img/lnu_etch.png} % Bakgrundsbild
    }
}
\newcommand\BackgroundPicLogo{
    \put(30,740){
    \includegraphics[keepaspectratio,scale=0.10]{img/logo.png} % Logga i övre vänstra hörnet
    }
}

% In the following you can change the title of your document
\title{	
\vspace{-8cm}
\begin{sidebar}
    \vspace{10cm}
    \normalfont \normalsize
    \Huge Assignment 2\\ % Dokumentets typ, t.ex. Examensarbete
    \vspace{-1.3cm}
\end{sidebar}
\vspace{3cm}
\begin{flushleft}
    \LARGE Computer Networks - an introduction\\ % Dokumentets rubrik
 %   \it \LARGE Examensarbete under arbete % Dokumentets underrubrik
\end{flushleft}
\null
\vfill
\begin{textblock}{6}(9,11)
\begin{flushright}
\begin{minipage}{\textwidth}
\begin{flushleft} \large
\emph{Author:} John Herrlin\\ % Författare
\emph{Email: } jh222jx@student.lnu.se\\
\emph{Author:} Rasmus Sjostrom\\ % Författare
\emph{Email: } rs222kp@student.lnu.se\\
%\emph{Handledare:} Dr.~Foo \textsc{Bar}\\ % Handledare
%\emph{Examinator:} Dr.~Mark \textsc{Brown}\\ % Examinator
\emph{Semester:} VT2016\\ % Termin
\emph{Area:} Computer Science\\ % Ämne
%\emph{Level:} G2F\\ % Nivå
\emph{Coursecode:} 1DV701 % Kurskod
\end{flushleft}
\end{minipage}
\end{flushright}
\end{textblock}
}


\date{} % Dagens datum, tomt i detta fallet. Använd \today för dagens datum.

\begin{document}
\pagenumbering{gobble}
\newgeometry{left=5cm}
\AddToShipoutPicture*{\BackgroundPic} % Lägger in backgrundsbild på första sidan
\AddToShipoutPicture*{\BackgroundPicLogo} % Lägger in LNU-logga på första sidan
\maketitle % Skriv ut titeln
\restoregeometry
\clearpage
%------------------------------------------------------------------------------
%	Svensk och engelsk version av abstract
%------------------------------------------------------------------------------
{

%------------------------------------------------------------------------------
\newpage
\pagenumbering{gobble} % Stäng av sidnumrering för innehållsförteckningssidan
\tableofcontents % Innehållsförteckning
\newpage % Ny sida
\pagenumbering{arabic} % Påbörja sidnumrering på 1

%------------------------------------------------------------------------------
% This is where you write the report:
\section{Introduction}

Code in module db and domain is not code that is accured in the assignment.
We created a small ORM becase the nature of POST and PUT requests.

Urls

In module handlers, we have a class called Urls.
If we can parse the request we put the request in Urls and we try to match the urls to different regexps.
If match we send the request to a View class that takes care of business logic.

The application starts in modue tcp, class TCPserver -> ServerThread.

\section{Problem 2}

The table in ~\ref{fig:responsecodesandusage} describes different types of response codes that the server sends back to the client.
We table describes response codes that are both in the G task in Problem 1 and VG-task 1 in Problem 2.
The reason to have everything in one table and not to split them up if for the simplicity and readability.

\begin{figure}[H]
  \centering  
  \begin{tabular}{ | l | l | }
    \hline			
    Code & Usage \\
    \hline			
    200 & OK, Many places where server handles response in a correct way. \\
    \hline			
    201 & Created, When success with POST on /post \\
    \hline			
    202 & Accepted, When /login success \\
    \hline			
    205 & Reset Content, When /logout success. Figure~\ref{fig:response205logout} \\
    \hline			
    400 & Bad Request, When NOT PUT on /put . Figure~\ref{fig:response400put} \\
    \hline			
    403 & Forbidden, When trying trying /post or /put but have not logged in (/login) before \\
    \hline			
    404 & Not Found, When trying to access a static file but cant find it OR if url doesn match any \\
    \hline			
    405 & Method Not Allowed, When NOT POST on /post \\
    \hline			
    500 & Server Error, If we cant parse the request \\
    \hline  
  \end{tabular}
  \label{fig:responsecodesandusage}
  \caption{Response codes and their usage.}
\end{figure}

\section{Screenshots}

hey 

\begin{figure}[H]
    \centering  
    \includegraphics[scale=1]{img/screenshots/response205logout.png}
	\label{fig:response205logout}
	\caption{Method: GET, Endpoint: /logout, Response: 205 Reset Content}
\end{figure}

\begin{figure}[H]
    \centering  
    \includegraphics[scale=0.6]{img/screenshots/response200login.png}
	\label{fig:response200login}
	\caption{Method: GET, Endpoint: /login, Response: 200 OK}
\end{figure}

\begin{figure}[H]
    \centering  
    \includegraphics[scale=0.6]{img/screenshots/response400put.png}
	\label{fig:response400put}
	\caption{Method: GET, Endpoint: /put, Response: 400 Bad Request}
\end{figure}




%------------------------------------------------------------------------------
%\bibliographystyle{IEEEtran}
% Here you will have to put your bibliography
%% \newpage
%% \bibliography{ieeedb.bib}{}
%% \bibliographystyle{IEEEtran}


\end{document}
